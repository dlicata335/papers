\subsection{Circle}

\paragraph{Formation rule}

\[
\infer{\ofglob\Gamma{\dsd{S^1}}{\typ i}}{}
\]

\paragraph{Introduction rules}
\[
\infer{\ofglob{\Gamma}{\base{1}}{\dsd{S^1}}}{}
\qquad
\infer{\ofglob{\Gamma}{\lp{1}}{\homj{\dsd{S^1}}{\base{1}}{\base{1}}}}{}
\]

\paragraph{Non dependent elimination rule}

\[
\infer{\ofglob{\Gamma,x:\dsd{S^1}}{\dsd{S^1\text{-}rec}_G(b, p; x)}{G}}
{\wfglob{\Gamma}{G}{n}{B} &
  \ofglob{\Gamma}{b}{G} &
  \ofglob{\Gamma}{p}{\homj{G}{b}{b}}}
\]
\FIXME{Is that a consequence of the dependent elimination rule?}
\\
\FIXME{Allow any dimension of glob or only a type? Given the situation for
  $\Pi$-types and products, it makes sense to have any dimension, and we will
  need it later, but the computation rule for $\lp{1}$ only makes sense in
  dimension 0… (we probably need an $\apn{}{}{}$ which can be applied to two
  globs)}

\paragraph{Non dependent computation rules}

\[
\begin{array}{rcl}
\dsd{S^1\text{-}rec}_{B}(b, p; \base{1}) & \equiv & b \\
\apn{1}{(x.\dsd{S^1\text{-}rec}_{B}(b, p; x))}{\lp{1}} & \equiv & p\\
\apn{n}{(x.\dsd{S^1\text{-}rec}_{B}(b, p; x))}
{\coh{\Delta.J}{\dsd{S^1}}{\delta}} & \equiv &
\coh{\Delta.J}{B}{\apn{n}{(x.\dsd{S^1\text{-}rec}_{B}(b, p; x))}{\delta}}
\\
\end{array}
\]

\paragraph{Dependent elimination rule}

\[
\infer{\ofglob{\Gamma,x:\dsd{S^1}}{\dsd{S^1\text{-}elim}_{x.G}(b, p; x)}{G}}
{\wfglob{\Gamma,x:\dsd{S^1}}{G}{n}{P} &
  \ofglob{\Gamma}{b}{G[\base{1}]} &
  \ofglob{\Gamma}{p}{\dhomj{x.G}{b}{b}{\lp{1}}}}
\]

\paragraph{Dependent computation rules}

\[
\begin{array}{rcl}
\dsd{S^1\text{-}elim}_{x.P}(b, p; \base{1}) & \equiv & b \\
\apn{1}{(x.\dsd{S^1\text{-}elim}_{x.P}(b, p; x))}{\lp{1}} & \equiv & p\\
\apn{n}{(x.\dsd{S^1\text{-}elim}_{x.P}(b, p; x))}
{\coh{\Delta.J}{\dsd{S^1}}{\delta}} & \equiv &
\depcoh{\Delta.J}{x.P}{\apn{n}{(x.\dsd{S^1\text{-}elim}_{x.P}(b, p; x))}
  {\delta}}
\\
\end{array}
\]
\FIXME{We need dependent Ap and dependent $\infty$-groupoid structure}

\subsection{Torus}

\paragraph{Formation rule}

\[
\infer{\ofglob\Gamma{\dsd{T}}{\typ i}}{}
\]

\paragraph{Introduction rules}
\[
\begin{array}{c}
\infer{\ofglob{\Gamma}{\baseT}{\dsd{T}}}{}
\qquad
\infer{\ofglob{\Gamma}{\pT}{\homj{\dsd{T}}{\baseT}{\baseT}}}{}
\\\\
\infer{\ofglob{\Gamma}{\qT}{\homj{\dsd{T}}{\baseT}{\baseT}}}{}
\qquad
\infer{\ofglob{\Gamma}{\fT}{\homj{(\homj{\dsd{T}}{\baseT}{\baseT})}
    {\pT\circ\qT}{\qT\circ\pT}}}{}
\end{array}
\]

$p\circ q$ is an abbreviation for the following (where the underscores can be
deduced from the context)
\[\coh{x:\diagbase,y:\diagbase,\alpha:\homj{\diagbase}xy,z:\diagbase,
  \beta:\homj{\diagbase}yz.\homj{\diagbase}xz}{\_}
{(\_,\_,q,\_,p)/(x,y,\alpha,z,\beta)}\]

\paragraph{Non dependent elimination rule}

\[
\infer{\ofglob{\Gamma,x:\dsd{T}}{\dsd{T\text{-}rec}_G(b, p, q, f; x)}{G}}
{\wfglob{\Gamma}{G}{n}{B} &
  \ofglob{\Gamma}{b}{G} &
  \ofglob{\Gamma}{p}{\homj{G}{b}{b}} &
  \ofglob{\Gamma}{q}{\homj{G}{b}{b}} &
  \ofglob{\Gamma}{f}{\homj{(\homj{G}{b}{b})}{p\circ q}{q \circ p}}}
\]

\paragraph{Non dependent computation rules}

\[
\begin{array}{rcl}
\dsd{T\text{-}rec}_{B}(b, p, q, f; \baseT) & \equiv & b \\
\apn{1}{(x.\dsd{T\text{-}rec}_{B}(b, p, q, f; x))}{\pT} & \equiv & p\\
\apn{1}{(x.\dsd{T\text{-}rec}_{B}(b, p, q, f; x))}{\qT} & \equiv & q\\
\apn{2}{(x.\dsd{T\text{-}rec}_{B}(b, p, q, f; x))}{\fT} & \equiv & f\\
\apn{n}{(x.\dsd{T\text{-}rec}_{B}(b, p, q, f; x))}
{\coh{\Delta.J}{\dsd{T}}{\delta}} & \equiv &
\coh{\Delta.J}{B}{\apn{n}{(x.\dsd{T\text{-}rec}_{B}(b, p, q, f; x))}{\delta}}
\end{array}
\]

\paragraph{Dependent elimination rule}

\[
\infer{\ofglob{\Gamma,x:\dsd{T}}{\dsd{T\text{-}elim}_G(b, p, q, f; x)}{G}}
{\wfglob{\Gamma,x:\dsd{T}}{G}{n}{P} &
  \ofglob{\Gamma}{b}{G[\baseT]} &
  \ofglob{\Gamma}{p}{\dhomj{x.G}{b}{b}{\pT}} &
  \ofglob{\Gamma}{q}{\dhomj{x.G}{b}{b}{\qT}} &
  \ofglob{\Gamma}{f}{\dhomj{x.(\dhomj{y.G}{b}{b}{x})}{p\circ q}{q \circ p}
    {\fT}}}
\]

\FIXME{Check that the previous rule is well-typed (need the typing rules for
  dependent equality)}

% \[
% \infer{\ofglob{\Gamma,x:\dsd{S^1}}{\dsd{S^1elim}_{x.P}(b, p; x)}{P}}
% {\ofglob{\Gamma,x:\dsd{S^1}}{P}{\typ i} &
%   \ofglob{\Gamma}{b}{P[\base{1}]} &
%   \ofglob{\Gamma}{p}{\dhomj{x.P}{b}{b}{\lp{1}}}}
% \]

\paragraph{Dependent computation rules}

\[
\begin{array}{rcl}
\dsd{T\text{-}elim}_{x.P}(b, p, q, f; \baseT) & \equiv & b \\
\apn{1}{(x.\dsd{T\text{-}elim}_{x.P}(b, p, q, f; x))}{\pT} & \equiv & p\\
\apn{1}{(x.\dsd{T\text{-}elim}_{x.P}(b, p, q, f; x))}{\qT} & \equiv & q\\
\apn{2}{(x.\dsd{T\text{-}elim}_{x.P}(b, p, q, f; x))}{\fT} & \equiv & f\\
\apn{n}{(x.\dsd{T\text{-}elim}_{x.P}(b, p, q, f; x))}
{\coh{\Delta.J}{\dsd{T}}{\delta}} & \equiv &
\depcoh{\Delta.J}{x.P}{\apn{n}{(x.\dsd{T\text{-}elim}_{x.P}(b, p, q, f; x))}
  {\delta}}
\end{array}
\]

\subsection{Binary product type}

\paragraph{Formation rule}

\[
\infer{\ofglob\Gamma{A\times B}{\typ i}}
{\ofglob\Gamma{A}{\typ i} &
  \ofglob\Gamma{B}{\typ i}}
\]

Products are negative, so we start by the elimination rules.

\paragraph{Elimination rules}
\[
\infer{\ofglob{\Gamma,u:A\times B}{\fst u}{A}}
{\ofglob\Gamma{A}{\typ i} &
  \ofglob\Gamma{B}{\typ i}}
\qquad
\infer{\ofglob{\Gamma,u:A\times B}{\snd u}{B}}
{\ofglob\Gamma{A}{\typ i} &
  \ofglob\Gamma{B}{\typ i}}
\]

\paragraph{Introduction rule}
\[
\infer{\ofglob{\Gamma}{(a, b)}{G}}
{\wfglob{\Gamma}{G}{n}{A\times B}
  & \ofglob\Gamma{a}{\Apn{n}{(x.\fst x)}{G}}
  & \ofglob\Gamma{b}{\Apn{n}{(x.\snd x)}{G}}
}
\]

\paragraph{Computation rules}

\[
\begin{array}{rcl}
  \apn{n}{(x.\fst x)}{(a,b)} & \equiv & a\\
  \apn{n}{(x.\snd x)}{(a,b)} & \equiv & b\\
  \coh{\Delta.J}{A\times B}{\delta} & \equiv &
  (\coh{\Delta.J}{A}{\delta},\coh{\Delta.J}{B}{\delta})
\end{array}
\]

\subsection{Torus to the product of circles}

\newcommand{\ttoc}{\mathrm{t2c}}
\newcommand{\ctot}{\mathrm{c2t}}
\newcommand{\PP}{\dsd{S^1\times S^1}}
\newcommand{\unitlr}{\mathrm{unitlr}}
\newcommand{\unitrl}{\mathrm{unitrl}}

We want to define a map $x:\dsd{T}\vdash\ttoc(x):\PP$. We use
$\dsd{T\text{-}rec}$:

\[\ofglob{x:\dsd{T}}{\dsd{T\text{-}rec}_{\PP}(b,p,q,f;x)}{\PP}\]

We need $b:\PP$, we take $b:=(\base{1},\base{1})$.

We need $p:\homj{\PP}{b}{b}$, we take $p:=(\lp{1},\id)$.

We need $q:\homj{\PP}{b}{b}$, we take $q:=(\id,\lp{1})$.

We need $f:\homj{(\homj{\PP}{b}{b})}{p\circ q}{q\circ p}$, so we need
$f:=(\alpha, \beta)$ where $\alpha$ is of type
\begin{align*}
  &\Apn{2}{(x.\fst x)}{(\homj{(\homj{\PP}{b}{b})}{p\circ q}{q\circ p})}\\
  &\equiv \homj{\Apn{1}{(x.\fst x)}(\homj{\PP}{b}{b})}
  {\apn{1}{(x.\fst x)}{(p\circ q)}}{\apn{1}{(x.\fst x)}{(q\circ p)}} \\
  &\equiv \homj{(\homj{\Apn{0}{(x.\fst x)}(\PP)}{(\apn{0}{(x.\fst x)}b)}
    {(\apn{0}{(x.\fst x)}b)})}
  {\apn{1}{(x.\fst x)}{(p\circ q)}}{\apn{1}{(x.\fst x)}{(q\circ p)}} \\
  &\equiv \homj{(\homj{S^1}{\base{1}}{\base{1}})}
  {\apn{1}{(x.\fst x)}{(p\circ q)}}{\apn{1}{(x.\fst x)}{(q\circ p)}} \\
  &\equiv \homj{(\homj{S^1}{\base{1}}{\base{1}})}
  {\apn{1}{(x.\fst x)}{(\lp{1}\circ\id{},\id{}\circ\lp{1})}}
  {\apn{1}{(x.\fst x)}{(\id{}\circ\lp{1}, \lp{1}\circ\id{})}} \\
  &\equiv \homj{(\homj{S^1}{\base{1}}{\base{1}})}
  {\lp{1}\circ\id{}}
  {\id{} \circ\lp{1}} \\
\end{align*}

We define
\[\unitrl(p):=\coh{x:\diagbase,y:\diagbase,\alpha:\homj{\diagbase}xy.
  \homj{\homj{\diagbase}xy}{p\circ\id{}}{\id{}\circ p}}{\_}
{(\_,\_,p)/(x,y,\alpha)}\]
\[\unitlr(p):=\coh{x:\diagbase,y:\diagbase,\alpha:\homj{\diagbase}xy.
  \homj{\homj{\diagbase}xy}{\id{}\circ p}{p\circ\id{}}}{\_}
{(\_,\_,p)/(x,y,\alpha)}\]

We can now define $\ttoc$ by

\[\ttoc(x):=\dsd{T\text{-}rec}_{\PP}((\base{1},\base{1}),(\lp{1},\id{})
,(\id{},\lp{1}),(\unitrl(\lp{1}),\unitlr(\lp{1}));x)\]

\subsection{Product of circles to the torus}

We want to define a map $u:\PP\vdash\ctot(u):\dsd{T}$. We first define a map
$x:\dsd{S^1}\vdash\ctot'(x):\dsd{S^1}\to\dsd{T}$ using $\dsd{S^1\text{-}rec}$:

\[\ofglob{x:\dsd{S^1}}{\dsd{S^1\text{-}rec}_{\dsd{S^1}\to\dsd{T}}(b,p;x)}
{\dsd{S^1}\to\dsd{T}}\]

We need $b:\dsd{S^1}\to\dsd{T}$, we take
$b:=\lambda{}y.\dsd{S^1\text{-}rec}_{\dsd{T}}(\baseT,\pT;y)$

We need $p:\homj{\dsd{S^1}\to\dsd{T}}{b}{b}$, so we need $p:=\lambda y.u$ where
\[\ofglob{y:\dsd{S^1}}{u}{\homj{\dsd{T}}
  {\dsd{S^1\text{-}rec}_{\dsd{T}}(\baseT,\pT;y)}
  {\dsd{S^1\text{-}rec}_{\dsd{T}}(\baseT,\pT;y)}}\]

We take $u:=\dsd{S^1\text{-}elim}_{y.\homj{\dsd{T}}
  {\dsd{S^1\text{-}rec}_{\dsd{T}}(\baseT,\pT;y)}
  {\dsd{S^1\text{-}rec}_{\dsd{T}}(\baseT,\pT;y)}}(b,p;y)$.

We need $b:(\homj{\dsd{T}}
{\dsd{S^1\text{-}rec}_{\dsd{T}}(\baseT,\pT;\base{1})}
{\dsd{S^1\text{-}rec}_{\dsd{T}}(\baseT,\pT;\base{1})})
\equiv(\homj{\dsd{T}}{\baseT}{\baseT})$, we take $b:=\qT$.

We need $p$ of the following type

\[\dhomj{y.\homj{\dsd{T}}
  {\dsd{S^1\text{-}rec}_{\dsd{T}}(\baseT,\pT;y)}
  {\dsd{S^1\text{-}rec}_{\dsd{T}}(\baseT,\pT;y)}}{\qT}{\qT}{\lp{1}}\]

It seems more or less reasonable to say that this glob reduces to the one
below. This will not easily scale to higher dimensions, though, but let’s
ignore this problem for now.

\begin{align*}
  &\homj{\dsd{T}}{\apn{1}{(y.\dsd{S^1\text{-}rec}_{\dsd{T}}(\baseT,\pT;y))}
    {\lp{1}}\circ\qT}
  {\qT\circ\apn{1}{(y.\dsd{S^1\text{-}rec}_{\dsd{T}}(\baseT,\pT;y))} {\lp{1}}}
  \\
  &\equiv \homj{\dsd{T}}{\pT\circ\qT}{\qT\circ\pT}
\end{align*}

We take $\fT$.

We have now defined $\ctot'$ by
\[\ctot'(x):=\dsd{S^1\text{-}rec}_{\dsd{S^1}\to\dsd{T}}
(\lambda{}y.\dsd{S^1\text{-}rec}_{\dsd{T}}(\baseT,\pT;y),
\lambda{}y.\dsd{S^1\text{-}elim}_{y.\homj{\dsd{T}}
  {\dsd{S^1\text{-}rec}_{\dsd{T}}(\baseT,\pT;y)}
  {\dsd{S^1\text{-}rec}_{\dsd{T}}(\baseT,\pT;y)}}(\qT,\fT;y);x)\]

Finally we define $\ctot$ by
\[\ctot(u):=(\ctot'(\fst u))(\snd u)\]

%%% Local Variables:
%%% mode: latex
%%% TeX-master: "paper"
%%% End:
